
Based on our specification we have created a design for the application's user interface (UI). The UI had to support the use cases specified in section \ref*{sec:use_cases}. In the MoSCoW analysis we specified a ranking of features from the use cases in order of \textit{must have}, \textit{should have},\textit{could have} and \textit{wont have}. In the design we prioritized the features listed in the \textit{must have} category. The two features \textbf{Students can retrieve exercises descriptions} and \textbf{Students can submit exercise attempts}, listed in the \textit{must have} category corresponds to the use case \textit{Solve exercises}.
In figure \ref{fig:wfExercise} the wire-frame design that satisfies the use case.
% primary use cases
% solve exercises
\begin{figure}[H]
	\includegraphics[scale=0.6]{WireframeSolveExercise.png}
	\centering
	\caption{wireframe for solving an exercise}
	\label{fig:wfExercise}
\end{figure}

For both images in figure \ref{fig:wfExercise} a text input field can be seen located on the left. This text input field is where the user can input code for submission. On the right side of the left-most image of figure \ref{fig:wfExercise}, a box can be seen containing two tabs. The currently selected tab of this image is the \textit{Instructions} tab which is where the instructions for the current exercise is displayed. On the right ride of the right-most image is a similar view, except that the \textit{Result} tab has now been selected. This tab shows the result of the submitted code, whether it passed the tests for the exercise and whether it compiled. This satisfies the requirements stated in the use case \ref{sec:hej}
% create exercises
\begin{figure}[H]
	\includegraphics[scale=0.6]{createProblem.png}
	\centering
	\caption{wireframe for creating a problem}
	\label{fig:wfProblem}
\end{figure}

% Structure of creating sessions and syllabi
\begin{figure}[H]
	\includegraphics[scale=0.6]{browseSyllabus.png}
	\centering
	\caption{wireframe for browsing a syllabus}
	\label{fig:wfSyllabus}
\end{figure}

% secondary
% view solutions
% account use case