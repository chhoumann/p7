\pdfbookmark[0]{English title page}{label:titlepage_en}
\aautitlepage{%
  \englishprojectinfo{
    P7 Programming Paradigms Practice Platform
  }{%
    Internet
  }{%
    Fall Semester 2022 %project period
  }{%
    cs-22-sw-7-09 % project group
  }{%
    %list of group members
    Christian Bager Bach Houmann\\
    Daniel Overvad Nykjær\\
    Ivik Lau Dalgas Hostrup\\
    Marco Klaustrup Justesen\\
    Patrick Frostholm Østergaard\\
    Rasmus Høyer Hansen
  }{%
    %list of supervisors
    Carlos E. Muniz Cuza
  }{%
   December 21, 2022
  }%
}{%department and address
  \textbf{Computer Science}\\
  Aalborg University\\
  \href{http://www.aau.dk}{http://www.aau.dk}
}{% the abstract
This report presents a web-based teaching platform with the goal of helping students learn Haskell in a classroom setting. The platform provides a user-friendly interface for creating and solving exercises, as well as a system for managing and organizing the exercises and tracking student progress. By making use of concurrency and queuing mechanisms, the platform can handle many students submitting solutions at once. The platform uses a role-based system for distinguishing user permissions such that lecturers and students can access different parts of the platform. Lecturers are able to create and edit exercises, and view the progress of all students. Students can view their progress and submit solutions to exercises. The platform has been evaluated in terms of usability and scalability.
}{% keywords
}

% \cleardoublepage
% {\selectlanguage{danish}
% \pdfbookmark[0]{Danish title page}{label:titlepage_da}
% \aautitlepage{%
%   \danishprojectinfo{
%     Rapportens titel %title
%   }{%
%     Semestertema %theme
%   }{%
%     Efterårssemestret 2010 %project period
%   }{%
%     XXX % project group
%   }{%
%     %list of group members
%     Forfatter 1\\
%     Forfatter 2\\
%     Forfatter 3
%   }{%
%     %list of supervisors
%     Vejleder 1\\
%     Vejleder 2
%   }{%
%     1 % number of printed copies
%   }{%
%     \today % date of completion
%   }%
% }{%department and address
%   \textbf{Elektronik og IT}\\
%   Aalborg Universitet\\
%   \href{http://www.aau.dk}{http://www.aau.dk}
% }{% the abstract
%   Her er resuméet
% }}
%