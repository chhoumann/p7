In the following section we will describe how the project was managed, which techniques were used, as well as which issues we encountered. Throughout we will reflect on the decisions that were made in this context.

\section{Task management}
One of the primary objectives in any project is to manage the tasks that are generated throughout the project. The goal is to have the ability to maintain an overview of what has been done, what needs to be done and how the tasks are distributed between developers. Additionally, project managers also need often want to be able to estimate how long tasks may take to complete and their difficulty. To accomplish this, projects may utilize several project management tools and techniques. In this section we will focus on the tools that we utilized, while in section \ref{sec:agile-dev} we will focus on the techniques that were utilized to manage the project.

\subsection{Github projects}
Initially we decided on using a tool called Github Projects. Github Projects is a project management tool that integrates issues, which is a text description of bugs and features that should be included in the project as well as pull requests into a board similar to a Kanban Board. This meant that we could create a board that was categorized into four different types: \textit{Todo}, \textit{In progress}, \textit{Ready for review} and \textit{Ready}, giving us a clear overview of what needed to be done, what was being worked on, what needed to be reviewed and what was done. All accessed from the same website. 

While this tool offers a lot of benefits for developers in terms of project management, the tool was unfamiliar to all of the group members, as no member had any experience with the tool. The lack of experience became a hurdle within a few weeks of the project starting. This was due to several factors, including an unwillingness to understand the tool, but the primary issue that we were also learning other tools, frameworks and even a new programming language. All of which were more important to the project and therefore learning Github Projects became a less of a priority. To the point that we completely stopped using it, despite its promises.
This led us to revert back to tools that we had previously used.

\subsubsection{Notion}
Notion is a platform that allows users organize resources and manage projects in the same space. The platform allows you to customize the layout of your work space in a way that best suits the team. One particular feature of Notion that we used in this project was the Kanban board. In the Kanban board we defined four categories, similar to what we had in Github Projects, in which we could place work items:

\begin{itemize}
    \item Not started.
    \item In progress.
    \item Review required.
    \item Done.
\end{itemize}

While Notion provides fewer features in terms of integration with Github, the tool was familiar to us and provided the functionality that we needed to manage the project.
Once we switched to Notion, planning and updating the tasks became a more active part of our daily work. This led to better productivity as we were now able to better maintain an overview of what each member of the group was doing and and how far they were with each task.

\section{Agile development approach} \label{sec:agile-dev}
Irrespective of which tools we used, the goal from the beginning of the project was to use a SCRUM-like approach to project management. 
Generally, SCRUM is a framework that helps teams and organizations generate value through flexible solutions to complex problems. SCRUM is an iterative and incremental approach to project work where the project is broken up into sprints. A sprint is timeframe, usually 1-4 weeks in length, in which the the teams tries to complete the work that has been selected from a product backlog.





%introduction
%github projects failed - pivoted to kanban board
%agile development
%ci/cd
%code reviews
%pair programming
%scrum