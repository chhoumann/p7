\section{This may or may not be where this should be}


Therefore the responsibility of the frontend can be split into two categories: 
\begin{itemize}
    \item \textbf{Lecturer}-based responsibilities.
    \item \textbf{Student}-based responsibilities.
\end{itemize}

\subsubsection*{Lecturer-based responsibilities}
If the signed in user is a lecturer the frontend provides the lecturer the ability to create syllabi, sessions, exercises and test cases for the exercises, as well as they ability to solve and submit exercises. The ability for the lecturer to solve and submit exercises was included to allow lecturers to try out the exercises and ensure that the test cases behave as expected before releasing the exercises to the students. Additionally, the lecturer will also have access to a dashboard that shows which students have completed which exercises for a given session, in order to track progress and .

\subsubsection*{Student-based responsibilities}
Conversely, if the signed in user is a student, the fronted provides the user the ability to view syllabi, sessions and exercises as well as the ability to attempt exercises. A student will also have access to an overview showing which courses are available to the student and which exercises have been completed. 