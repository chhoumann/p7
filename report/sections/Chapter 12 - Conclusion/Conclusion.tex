The main purpose of this project was to create an internet-based application that would solve a problem. The developed system also needed to be tested. 
In this project, we developed a system for lecturers of the \textit{Programming Paradigms} course. 
This would help them conduct the course, allowing them to define exercises with accompanying tests that students can solve. 
The students can then receive fast feedback without the need for a teaching assistant to verify if their solution is correct. 

To create this application, we had to develop a working UI that the students and lecturers could use in order to interact with the system. The UI was developed to be user-friendly and provide a minimal amount of hindrance to solving exercises.
A backend was developed to serve the frontend to the users, and answer requests from the frontend. 
The requests would then be rerouted to the relevant components of the system. 
In order to store information regarding users and exercises, a database was implemented.
To be able to give the users feedback on their submitted exercises, we had to implement a component that would run the accompanying tests for an exercise on the submitted program. 
After the test runner has run the tests, it will send back the result to the user.
In order to ensure that the system would be scalable, we performed a stress test on different versions of the program. 
The stress tests showed that ... \todo{Insert stress test result}
To confirm that the UI had high usability, we performed a usability test based on the system usability scale.
The test was performed on students who at the time were taking the \textit{Programming Paradigms}\todo{add textit to Programming Paradigmes in whole report} course.
On average the SUS score was 82,5 which is close to what is considered excellent.  

In this project we set out to create a scalable system, that would be an improvement to the \textit{Programming Paradigms} course.
Our application can deliver swift evaluation of students' exercise submissions, and based on our tests it will be scalable up to a whole class of students.
