\chapter{Benchmarking} \label{chap:Benchmarking}
To ensure that the developed platform can handle an appropriate amount of users, the Test Runner has been stress tested and benchmarked.
In this chapter we present how these tests and benchmarks have been conducted, and describe the implementation. To conduct the benchmarks, we have utilized BenchmarkDotnet (see \ref{chap:preliminaries}) and defined three components used solely for the benchmarks:
a client component simulating possible Test Runner client behavior, a Benchmark component responsible for orchestrating benchmarks for different Test Runner versions, and a test component ensuring that the client component can contact the Test Runner before running the benchmarks.

\section{Orchestration and Test Runner parameterization}
The benchmark and test components are managed in their own docker containers, and are connected to a network connecting them to a Test Runner container. 

\todo{create a picture for the orchestration}
\section{Simulating client behavior}


The test project consists of three components: clients, tests, and benchmarks.
The client component provides implementation for client behaviors. Different clients provide different ways of contacting the Test Runner.

The client project detail different client behaviors
The benchmark contain different benchmarks
The test contains test that ensure that the Test Runner can be contacted before running the benchmark.