\section{Queue System} \label{sec:queue-system}
We have two endpoints that need to share data, and the Rust programming language does not allow global variables.
Therefore, we opted to create a shared state data structure using the \texttt{layer} feature in the axum framework.
The \texttt{layer} feature is axum's default implementation of middleware for processing data across a group of endpoints.
In our case, there are two key variables inside the shared state structure that both endpoints need access to:
\begin{enumerate}
    \item \textbf{A sender channel:} In practice, this channel is used as a worker queue in which code submission requests are enqueued for processing. It has a corresponding \textbf{receiver channel}, which is to dequeue and schedule work. Importantly, requests sent via the sender channel are received in a first-in-first-out order.
    \item \textbf{The job results hashmap:} Upon first receiving a code submission request, a UUID is generated. This is used as a key inside the job results hashmap, where the value is initially set to the Rust \texttt{Option} value of type \texttt{TestRunnerResult} --- in other words, a key-value pair with the UUID referencing an empty job result, indicating that the work has not yet been completed. An \texttt{Option} type is simply a nullable data structure in Rust; an optional value. When the Test Runner has processed the code submission, the result is inserted as the value of the \texttt{Option} type. This value can then be looked up when the client requests the result processing their code submission, since the client is provided with the UUID upon sending a code submission, as previously mentioned.
\end{enumerate}

In order to limit the number of GHC interpreter processes, a thread pool with a specified size is created whose responsibility is to process code submissions.
These threads have access to the aforementioned receiver channel containing work to be processed.
The creation and content of the the thread pools can be seen in code snippet \ref{lst:worker-thread}.

\begin{lstlisting}[language=rust, escapechar=~, caption={Rust code showing allocation of thread pools and scheduling of code submission processing}, label={lst:worker-thread}]
async fn worker_thread(
    rx: Receiver<TestRunnerWork>,
    job_results: Arc<Mutex<Box<HashMap<Uuid, Option<TestRunnerResult>>>>>,
    limit: usize
) {
    let stream = tokio_stream::wrappers::ReceiverStream::new(rx);

    stream.for_each_concurrent(limit, |work| async {
        debug!("Running worker thread on UUID {}...", work.id); ~\label{line:concurrent-start}~

        let res = schedule_test(work.submission).await;
        let mut map = job_results.lock().unwrap();
        map.insert(work.id, Some(res)); ~\label{line:concurrent-end}~
    }).await;
}
\end{lstlisting}

We use the \texttt{futures} crate in Rust to allocate some number of threads dictated by the variable \texttt{limit} of type \texttt{usize}.
The receiver channel is converted to a \texttt{ReceiverStream} using the \texttt{tokio\_stream} utility crate.
A \texttt{ReceiverStream} is simply a wrapper which implements the \texttt{Stream} trait from the \texttt{futures} crate.
This allows us to combine functionality from the two crates.
In practice, implementations of the \texttt{Stream} trait simply provide a stream of values asynchronously --- essentially an asynchronous iterable.

Using this, we can call \texttt{for\_each\_concurrent} from the \texttt{futures} crate to run a number of tasks asynchronously but with the aforementioned limit.
The \texttt{futures} framework then manages the thread pool automatically, and we only need to specifiy the work that the threads need to do, which can be seen in lines \ref{line:concurrent-start} --- \ref{line:concurrent-end}.

The \texttt{schedule\_test} function calls the Test Runner, which is responsible for running the GHC interpreter and returning the results of running the tests on the submitted code from the client, as described in chapter \ref{chap:TestRunner}.
Once the result is computed, the thread acquires the mutex lock to the hashmap or blocks until it can acquire the lock.
It then inserts the result into the hashmap by using the UUID as the key.
Once the \texttt{map} variable goes out of scope, the mutex is unlocked so other threads may access the hashmap.

In summary, a specified number of concurrent threads in a thread pool continuously receive code submissions from the \texttt{ReceiverStream} channel.
Each submission is then processed by one of these threads, and the result of the Test Runner is packaged into the hashmap where the UUID serves as a key to look up the result.

While this approach solved the initial problem of only being able to process one code submission at a time, a new problem is the fact that the hashmap continuously grows over time --- data is only ever added to the hashmap and never removed.
Consequently, the memory use of the Test Runner would eventually exceed its capacity, resulting in a crash.
In the following section, we introduce a sweeping system as a solution to this problem.