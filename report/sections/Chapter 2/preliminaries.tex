\chapter{Preliminaries}
The following chapter contains short introductions to the most important technologies used in the repport. 

\section{Glasgow Haskell Compiler} %-- Haskell afsnit istedet??
Out of the three currently available Haskell compilers available the Glasgow Haskell Compiler (GHC) is used on the backend for interpreting the Haskell exercises and running the test suite. GHC was chosen over the other up-to-date compilers due to it being the standard compiler recommended by the Haskell Organization \cite{Haskell_GHC} as well as it being the compiler used in the Haskell Course.

\section{Hspec}
To enable running Haskell tests on the backend we needed a test framework that allowed us to define tests for the exercises. 

Here Hspec was chosen over Hunit since it allows the professor the define the tests in an easy-to-understand domain-specific language(DSL). 
This makes the framework easier to work with as well as gives us the option to show some of the tests to the users to give them a more specific idea of what is expected from the exercise.

Hspec allows integration with other Haskell test frameworks if needed and automatic discovery of test files. 
Furthermore, Hspec also allows for parallel test execution which is preferable when trying to optimize for larger numbers of users.

\begin{lstlisting}[language=CSharp, caption={An example of a Hspec Test.}, label={lst:HspecTestExample}]
{
 module moduleNameSpec (spec, main) where

 import Test.Hspec
 import moduleName (functionName)
 
 spec = Spec
 spec = do
 	describe "Function Name" do
 	it "functionality description" $ do
	functionname x `shouldBe' y
}
\end{lstlisting}

\section{Rust}
Rust is a multi-paradigm, general-purpose programming language originally developed by Graydon Hoare in 2006 as a personal project at Mozilla. 
The project was then sponsored by Mozilla from 2009 to 2021 where the project along with all trademarks was moved to the newly create Rust Foundation. 

The idea behind Rust was to challenge the idea that high-level ergonomics and low-level control come at the cost of each other\cite{Rust_Book}.
Most other programming languages today operate in a band where either safety or control is valued highest with the other diminishing the higher you prioritize the first. Such as C and its high level of control and lack of safety. 
Here rust tries to provide the low-level power of C along with the solid safety features programmers have gotten used to in high-level languages like Python. 
As Such the main overall goal of Rust is to provide as much safety as possible in a low-level programming language without sacrificing any speed\cite{Rust_in_Action}.

In order to achieve this goal Rust provides a large number of safety features among these is its most distinguishing feature which prevents invalid data access at compile time. 
Considering a research article by Microsoft Security Research Center stating that invalid data access is behind 70\% of serious security bugs in modern programs this is a large deal\cite{Safe_Systems_Languages}. 
Instead of relying on the developer to deal with this, the Rust compiler guarantees any compiled program is memory-safe.


\section{Docker}
In order to set up a local development environment for the used APIs and Databases, the open-source containerization platform Docker was chosen. 

The main purpose of Docker is to simplify deliveries in distributed systems with the use of so-called containers\cite{Docker_Container}.
Containers are bundles of code and required dependencies. This allows developers to move around prepackaged applications ensuring that the developed software runs the same regardless of what system is use\cite{Docker_Container}.

The containers themselves are created from an image. Here the image is the executable package that encapsulates everything required in order to run the application. On runtime, the image is then turned into our container. 
This is possible due to the Linux kernel's build process isolation and virtualization capabilities. These also allow a single host system to share its resources with multiple application components in the same way a hypervisor enables the resources of a single computer to be shared between multiple virtual machines\cite{Docker_Container}.

\section{tRPC}
TRPC is created in order to solve the paint point API presented in modern best practices for web programming using TypeScript. 
Here an improved way of statically typing endpoints and sharing them between clients and servers is needed\cite{tRPC}.

TRPC is a library that enables developers to build type safe APIs without having to deal with schemas or code generation. 
It is structured in a way that allows for types to be shared between the client and server by only important the types instead of the actual server code. 
This in turn allows developers to not have the actual server code exposed in the \frontend{} 
Furthermore, tRPC is created in a way to fully take advantage of the powers from a full-stack TypeScript which a language-agnostic framework like GraphQL cannot\cite{tRPC}.

\section{Prisma}
The main goal behind Prisma is to increase developers productivity by providing a clean and type-safe API for submitting database queries\cite{Prisma_Why}.
Here Prisma tries to avoid to large tradeoffs in productivity that comes with increased control in the main database tools currently in use for TypeScript. 


\section{Postgres (or other db)} %Prisma?

