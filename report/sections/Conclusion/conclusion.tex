The main goal of this project was to create an internet-based application that would provide students participating in the Programming Paradigms course with swift evaluation of their proposed exercise solutions.
The system allows lecturers to create exercises for students with accompanying tests for verifying correctness of student solutions.

To create this application, we developed a user interface that students and lecturers can use in order to interact with the system. The user interface and an accompanying server for handling user interactions.
To enable persistance of users and their interactions with the system, a login system was implemented using a layered architecture.
Information regarding the users, their submitted exercise solutions, and exercises and accompagnying tests was persisted on a Postgres database.
The user interface was developed with user-friendliness in mind, providing minimal hindrance in accessing and solving exercises.
A backend was developed in order to process and serve frontend requests.
The backend is connected to the Test Runner. The Test Runner uses the GHC interpreter to run tests on exercise submissions to provide users with feedback on their submission.
After these tests have been run, the results are accessible by polling the Test Runner.
We implemented a queue system to avoid too many running instances of the GHC interpreter process --- it stores code and test data from requests for future processing if all worker threads are occupied.

In order to ensure that the system is scalable, we performed benchmarks by measuring response times during stress testing the Test Runner with varying configurations.
Based on these benchmark results, the application will be able to deliver results within an acceptable time for at least up to 100 concurrent requests; good enough for a classroom setting.
To ensure that the developed user interface is intuitive for users, we performed a usability test.
The test was performed on students who, at the time, were taking the \textit{Programming Paradigms} course.
According to the users from our evaluation, the developed application has a close to excellent UI.
They also remarked that they would have liked to use our system throughout the \textit{Programming Paradigms} course, further suggesting that our solution fulfills the requirements presented in the problem statement.

In conclusion, the application we have developed can deliver swift evaluation of student exercise submissions through an intuitive and coherent user interface.