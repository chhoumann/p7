The main goal of this project was to create an internet-based application that would solve some problem.
Furthermore, another requirement in the study regulation is that the developed system also needs to be tested for scalability. 
In this project, we developed a system for lecturers of the \textit{Programming Paradigms} course that can provide fast and efficient feedback on student exercise submissions.
The system allows lecturers to create exercises for students and define accompanying tests for verifying correctness of student solutions.

To create this application, we developed a working user interface that students and lecturers can use in order to interact with the system. 
The user interface was developed with user-friendliness in mind, providing minimal hindrance in accessing and solving exercises. 
A backend was developed in order to process and serve frontend requests. 
Supplementing this, a database was also implemented to store information regarding users and exercises.
The backend features a component that uses the GHC interpreter to run tests on exercise submissions to provide users with feedback on their submission.
After these tests have been run, the results are available to the user through polling.
We implemented a queue system to avoid too many running instances of the GHC interpreter process --- it stores requests for future processing if all worker threads are occupied.

In order to ensure that the system is scalable, we performed stress tests on different versions of the program with varying configurations.
Based on these tests, the application will be able to deliver results within an acceptable time for at least up to 100 concurrent requests; good enough for a classroom setting. 
To ensure that the developed user interface is intuitive for users, we performed a usability test.
The test was performed on students who, at the time, were taking the \textit{Programming Paradigms} course.
According to the users from our evaluation, the developed application has a close to excellent UI. 

In summary, the application we have developed can deliver swift evaluation of students' exercise submissions through an intuitive and coherent user interface. 
Finally, the users from the usability test remarked that they would have liked to use our system throughout the \textit{Programming Paradigms} course, further suggesting that our solution fulfills the requirements presented in the problem statement.