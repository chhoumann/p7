\section{Tokens and Polling}
To prevent connection timeouts, the Test Runner web server sends back a unique token to the client immediately upon receiving a POST request.
This token serves as a ticket for the client.
Using the token, the client can continuously poll the server for result of processing the code submission.
We implemented a new endpoint which expects the token as a parameter.
There are three possible responses from this endpoint:
\begin{enumerate}
    \item If the given token is not present within the queue, the response body contains a special "not found" message.
    \item If the given token is present within the queue but has not yet been processed, the response body contains a special "in progress" message.
    \item If the given token is present within the queue and has been processed, the response body contains a special "complete" message as well as the results of running the tests on the submitted code from the request.
\end{enumerate}

In summary, the Test Runner web server features two HTTP endpoints.
The first endpoint expects a POST request containing a code submission which it enqueues into a worker queue along.
It then responds to the client with a unique token (UUID).
This UUID is also used by the web server as a key to look up the given code submissions in the queue.
The second endpoint expects a GET request with a UUID as a parameter used to look up the result of processing the code submission with the matching UUID.
It then responds to the client with the result of processing that submission if it was found.

With the connection timeout problem solved, we needed a way to limit the number of running GHC interpreter processes.
We did this by implementing a queue system, which we describe in the following section.