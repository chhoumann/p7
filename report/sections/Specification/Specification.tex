This chapter presents an overview of our system using a system specification containing the purpose and the scope of our system, the target users, the constraints of the project, the use cases, and the feature prioritization using a MoSCoW analysis.

\section{Purpose and Scope of the System}
The purpose of our system is to provide a better approach to learning functional programming in Haskell using a software application.
We aim to make a system where students can solve exercises. The system should be able to provide feedback on whether the user's solution fulfills the requirements of the given exercise.
In addition, the lecturer should be able to add exercises and write specifications for exercise solutions.  
Currently, 172 students attend the course.
We will use the number of course attendees to create an estimation of an operational profile which will be used to stress test the solution, as described in \ref{chap:Benchmarking}

\section{Target Users}
The system will have two main users: The lecturer that will conduct the course, and the university students enrolled in the course.
Therefore, the responsibility of the frontend can be split into two categories: \textbf{Lecturer}-based responsibilities and \textbf{student}-based responsibilities.


\subsubsection*{Lecturer-based responsibilities}
The platform provides the lecturer the ability to create syllabi, sessions, exercises, and test cases for the exercises, as well as the ability to solve and submit exercises. 
The lecturer also has access to a dashboard, allowing them to track student progress on exercises.

\subsubsection*{Student-based responsibilities}
The \frontend{} gives the student the ability to view syllabi, sessions, and exercises as well as the ability to attempt exercises. A student will also have access to an overview showing which courses are available and which exercises have been completed.

\section{Constraints of the Project}
Since this project is developed as a semester project at \aau{}, there will be some constraints that we must adhere to.
One of the constraints is the time constraint: we have a single semester, or approximately three months, to develop the system.
In addition, the study regulation requires that the system be internet based system and scalable.

\section{Use Cases} \label{sec:use_cases}
The goal of the project is to improve the teaching methods in the \textit{Programming Paradigms} course.
In the following, we describe use cases that we deem necessary to fulfill in order for the system to improve the course.
These use cases are presented in no particular order.

\begin{enumerate}
    \item \textbf{Use case: Accounts}\newline
    To keep track of what exercises each user has solved, and which permissions they have, a user should be able to log in to the platform, and be registered as a student or a lecturer.
    \item \textbf{Use case: Solve exercises}\newline
    A Student should be able to choose an exercise session and a specific exercise. The exercise description is then retrieved and shown to the user.
    As the user completes the exercise, they should also be able to submit their solution, and verify that it fulfills the requirements of the exercise. If this is not the case, an error message should be displayed to the user describing the problem.
    
    \item \textbf{Use case: Create exercises}\newline
    A lecturer should be able to create and edit syllabi.
    Within a syllabus, the lecturer should also be able to create and edit exercise sessions and included exercises.
    For each exercise, the user should be able to specify an exercise description and tests that must pass, the tests should be performed on the submitted code. Furthermore, the lecturer should be able to specify template code present within an exercise immediately upon opening it.
    
    \item \textbf{Use case: Show statistics over exercises} \newline
    A lecturer should be able to view statistics about exercises.
    This includes how many attempts were used by the students to complete it or the time taken.
    Using this, the lecturer can determine if the difficulty is reasonable.

    \item \textbf{Use case: View solutions}\newline
    The lecturer can view all submitted solutions to see if students use the correct approach.
    Students should be able to view their own solutions and previous attempts for each exercise.

    \item \textbf{Use case: Best solutions}\newline
    The lecturer should be able to select code solutions submitted by students and publish them to all other students, allowing them to access solutions discussed during sessions.

\end{enumerate}
 

\section{MoSCoW} \label{sec:MoSCoW}
We need a way to prioritize the previously described use cases to gain an overview of which features the solution must, should, could and will not have.
To do this, we perform a MoSCoW analysis.
Each feature will be put into a category based on how important it is for the system.
Table \ref{tab:MoSCoW} shows the feature prioritizations for a working solution.
The \textit{Must have} category includes features required for students to solve exercises.
The \textit{Should have} category includes additional features needed to fulfill the \textbf{Solve exercises} use case as well as the \textbf{Create exercise} use case.
They are categorized as \textit{Should have} since they constitute the primary features of a functional teaching platform.
Account handling and the ability to access previous submissions are placed in the \textit{Could have} category, as they are not needed to provide a usable solution, but would be useful to have in a practical setting.
Lastly, statistics and solution highlighting are placed in the \textit{Won't have} category, as they have been deemed out of scope following to the constraints of the project.


% Please add the following required packages to your document preamble:
% \usepackage[table,xcdraw]{xcolor}
% If you use beamer only pass "xcolor=table" option, i.e. \documentclass[xcolor=table]{beamer}
\begin{table}[H]
    \begin{tabular}{|l|llll}
    \cline{1-1}
    \cellcolor[HTML]{C0C0C0}\textbf{Must have}                                     &  &  &  &  \\ \cline{1-1}
    Students can retrieve exercise descriptions                                    &  &  &  &  \\ \cline{1-1}
    Students can submit solution attempts                                          &  &  &  &  \\ \cline{1-1}
    The system can interpret solution attempts                                     &  &  &  &  \\ \cline{1-1}
    Lecturers can group exercises into sessions                                    &  &  &  &  \\ \cline{1-1}
    \cellcolor[HTML]{C0C0C0}\textbf{Should have}                                   &  &  &  &  \\ \cline{1-1}
    Lecturers can create tests for exercises                                       &  &  &  &  \\ \cline{1-1}
    The system can verify whether an exercise submission passes each test          &  &  &  &  \\ \cline{1-1}
    Users can login to determine if they are a student or a lecturer               &  &  &  &  \\ \cline{1-1}
    \cellcolor[HTML]{C0C0C0}\textbf{Could have}                                    &  &  &  &  \\ \cline{1-1}
    Students can access previous exercise attempts                                 &  &  &  &  \\ \cline{1-1}
    Lecturers can view all exercise attempts submitted by students                 &  &  &  &  \\ \cline{1-1}
    \cellcolor[HTML]{C0C0C0}\textbf{Won't have}                                    &  &  &  &  \\ \cline{1-1}
    Lecturers can share submitted solutions so they are viewable by other students &  &  &  &  \\ \cline{1-1}
    Lecturers can access statistics for exercise attempts                          &  &  &  &  \\ \cline{1-1}
    \end{tabular}
    \caption{\label{tab:MoSCoW}MoSCoW Analysis}
\end{table}

The \textit{Must have} category from the MoSCoW analysis represents the minimal amount of features to satisfy the project goals. Throughout the report, we will refer to this as the Minimal Viable Product (MVP).
In the following chapter, we will describe the design of the system, which is based on the MVP.
