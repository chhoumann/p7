\section{Minimal Viable Product}
Considering that students use a variety of different operating systems, a web application would be the best way to ensure easy portability.
To facilitate the problems mentioned in \ref{sec:initial-problem}, we imagine a website that allows users to complete the following tasks:
\begin{itemize}
	\item The professor should be able to create a syllabus with exercises that students are expected to solve
	\begin{itemize}
		\item The professor should be able to create multiple syllabi where students can self-enroll
	\end{itemize}
	\item The professor should be able to track the amount of exercises solved by each enrolled student
	\begin{itemize}
		\item The professor should be able to view additional statistics like the number of attempts used by a student to solve an exercise
	\end{itemize}
	\item The professor should be able to specify criteria in order to pass each exercise 
	\begin{itemize}
		\item Students should be able to verify whether their solution is correct
		\item Verifying solutions should be automatically handled by the system
	\end{itemize}
	\item Students should be able to collaborate on the same code
	\item Students should have access to discussed solutions later on
	\item The professor should be able to see each student's code
\end{itemize}

We then prioritized the presented features required to construct a minimal viable product (MVP).
We plan on using this in practice to see whether it serves as a useful solution to the presented problems. 
The MVP is meant to be used as a demo product for gathering additional feedback.
As such, only a minimal number of features are included in our proposed solution:

\begin{displayquote}
A website that allows a professor to specify a set of exercises as well as the completion criteria for each of these exercises. Furthermore, a way to verify whether submitted code is a correct solution to the given exercise.
\end{displayquote}