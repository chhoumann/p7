\section{Problem analysis}
In this section we will analyze the problem area, as well as existing solutions. 

\subsection*{Screen sharing by AAU}
Currently AAU uses a room designed by \textit{Center for Digitally Supported Learning} (CDUL) to support interactive learning. 
The room is set up with a monitor at each table where a group will sit. The group can then share their screen on the monitor, and the lecturer can then share their own screen or the screens of a group of students to the entire class on the monitors. 
Typically the content that is shared with the class is groups solutions to exercises, or problems that many had difficulties with.
The advantage of this is that many of the common issues are explained to the entire class.
The exercises consist of a mix of pair programming and group exercises, which encourages collaboration and sparing.
In the room during exercise sessions there are teaching assistants and the professor, which will answer the students questions and help if they are stuck on a problem. 

\todo{insert picture of table from PP}

However, with this way of teaching new problems arise.
Since students use their own local development environments for programming exercises, other students cannot revisit the discussed exercise solutions created by other groups. This makes it difficult to reflect on and learn from other students' solutions.
Furthermore, it is difficult to verify whether ones own solution is correct and lives up to the requirements of the exercise.
In fact, it is not uncommon for people to misunderstand an exercise and therefore implement an incorrect solution.

\subsection{Initial Problem} \label{sec:initial-problem}
The restructuring of the course already seems like a major improvement over the former structure, but as mentioned it still suffers from a few notable problems related to the execution of the exercise sessions:
\begin{itemize}
	\item No way to revisit discussed solutions
	\item No way to verify that your own solution is correct
	\item No way for the entire group to work together on the same code
	\item No way for the professor to verify how many exercises the students have solved
	\begin{itemize}
		\item Difficult to verify that all students have solved the problems
		\item Difficult to judge the skill level of the students and whether the difficulty of exercises is suitable
	\end{itemize}
\end{itemize}

\subsection{current platforms}
Taking a look at the current market some of the most popular platforms include: codewars, coursera and code academy.
Here we will describe these platforms, how they work, and what their pros and cons are.

\subsubsection*{codewars}
Codewars is a platform for solving programming exercises, where the goal is to create the best solution possible for a given problem. It does not include guides or tips for exercises, making it hard for new programmers, or programmers unfamiliar with the used language, to get into. If the programmer do not know how to solve the problem they are forced to search for information elsewhere to solve it. The only feedback that is given on a solution is given in the form of user votes. After the problem has been solved it is however possible to view other peoples solutions. 
Codewars would not be ideal to be used by a university lecturer, as codewars is not made to be used as a teaching environment, and more used as a practice environment.

\subsubsection*{coursera}
Coursera is a learning platform making on demand lectures which focus on mastering skills used in ones career. 
The courses are created by third parties, which consist of universities and relevant companies. 
Courses include exercises, which are graded, but other feedback is from ones peers. If a programmer gets stuck on an exercise, there is no one they can ask, and they are necesitated to find information in the video lectures, or on another site.
Coursera would not be ideal to be used by a university lecturer as exercises do not have instant feedback on submitted assignments.

\subsubsection*{codeacademy}
Codeacademy is a learning platform where all of the material is in text form, and exercises for the material. The intended format is that the material is read followed by the exercises. The only feedback that is received is based on whether the tests for the submitted code passes. It is instant, but no feedback is given on coding style, best practices and efficiency. 
They have their own curriculum, and would not be ideal to be used by a university lecturer as they would not be able to use their own curriculum, outside of codeacademy. It is also not possible to compare your solutions to other submitted solutions.

\subsection{Problem statement}
    
Current platforms for learning to program are mainly split into three types: Coding exercises, video guides, online curriculum. 
Each of these types of learning platforms has their own advantages and disadvantages. AAU could benefit from an application which has been designed specifically to the case of teaching at AAU, since it is different from how programming is taught on the internet
To adress the shortcomings of the current system used during exercises, an application designed to facilitate fast concrete feedback on exercises, for the students and the lecturer, could be created.
By summarizing the main problems following the switch to the new teaching approach, an initial problem statement is written.
\begin{displayquote}
How can one ensure a mutual feedback loop between students, teaching assistans and the professor during exercise sessions?
\end{displayquote} 

